\documentclass{aiaa-tc}

%\usepackage[margin=1.0in]{geometry}
\usepackage{fullpage}
\usepackage{graphicx}
\usepackage{bm} %required for bold in math mode for greek symbols
\usepackage{amsmath} %for bmatrix
\usepackage{amsfonts} %for math script font
\usepackage{url} %for website citations

\usepackage[space]{grffile} %for filepaths with spaces
\usepackage[mathscr]{euscript} % for scripty L1 symbol

%define degree symbol:
\newcommand{\degree}{\ensuremath{^\circ}}

\newcommand{\fr}[1]{$#1^+$} %command to write a reference frame
\newcommand{\br}[2]{[#1]_{#2}} %bracket operator with subscript
\newcommand{\tvect}[3]{\begin{bmatrix}#1\\#2\\#3\end{bmatrix}}% 3 x 1 vector
\newcommand{\tvecth}[3]{\begin{bmatrix}#1&#2&#3\end{bmatrix}}% 1 x 3 vector
\newcommand{\B}[1]{\textbf{#1}} %bold for regular vectors
\newcommand{\U}[1]{\hat{\textbf{#1}}} %hats and bold for unit vectors
\newcommand{\BG}[1]{{\bm #1}}           % for vectors using greek letters
\newcommand{\ddt}[1]{\frac{d#1}{dt}} %for time derivatives
\newcommand{\ddarg}[2]{\frac{d#1}{d#2}} % for general derivatives
\newcommand{\pparg}[2]{\frac{\partial#1}{\partial#2}} % for general derivatives
\newcommand{\kron}{\otimes} %redefines \kron to produce kronecker product symbol, for convenience
\newcommand{\squig}[1]{\ensuremath{[{}^{\times} #1]}} % cross product matrix operator
\newcommand{\Lone}{\ensuremath{{\mathscr{L}_1}}} % L1 w/ scripty L


\title{ 3D relative attitude cooperative estimation }
\author{Tim Woodbury}

\let\endtitlepage\relax %surpress line break after title page

\begin{document}

\maketitle

The problem of relative attitude estimation between cooperating agents, with no a priori knowledge of the initial state or known common features, is considered. As a feasability check, the simplest possible scenario is considered: each agent has measurements from a gyroscope and a vector to the other agent. Under these circumstances, can the relative attitude be estimated in a decentralized sequential estimation framework? Estimation of relative attitude is considered essential to cooperative estimation, since otherwise agents cannot use measurements made by another agent without coupling the state estimates of the two agents.

The governing equations between an agent, designated $i$, and a remote agent $j$ that $i$ senses, are now developed. We consider the problem of estimating the attitude of $j$ relative to $i$, parameterized by generic coordinates $\B{q}_{j/i}$. The angular velocities of each agent relative to the inertial frame are $\BG{\omega}_{i/n}$ and $\BG{\omega}_{j/n}$, and each are assumed to be coordinatized in the corresponding agent body-fixed frame when applicable. The time rate of change of the attitude coordinates is simply linear function of the relative angular velocity $\BG{\omega}_{j/i}$, where $[A(\B{q}_{j/i})]$ is a matrix of appropriate dimensions that depends on $\B{q}_{j/i}$ only:

\begin{equation}
\dot{\B{q}}_{j/i} = [A(\B{q}_{j/i})]\BG{\omega}_{j/i}
\label{eq:qji}
\end{equation}

The relative angular velocity in the \fr{j} frame is simply:

\begin{equation}
[\BG{\omega}_{j/i}] = [\BG{\omega}_{j/n}] + [C_{j/i}(\B{q}_{j/i})][\BG{\omega}_{i/n}]
\label{eq:omegaji}
\end{equation}

The unit vector pointing from $i$ to $j$, i.e. the vector measured by agent $i$, is designated $\U{r}_{j/i}$. Again, letting the measurement by each agent be coordinatized in its own body-fixed reference frame, the measurements made by $i$ and $j$ of onw another should be related by:

\begin{equation}
[\U{r}_{i/j}] = -[C_{j/i}(\B{q}_{j/i})][\U{r}_{j/i}]
\label{eq:rijtemp}
\end{equation}

Subtracting the left-hand side of Eq. \ref{eq:rijtemp} from the right-hand side, the measurement equation is obtained. Note that the measurement $\tilde{\B{y}}$ is a function of the estimated attitude $\hat{\B{q}}_{j/i}$:

\begin{equation}
\tilde{\B{y}} = -[C_{j/i}(\hat{\B{q}}_{j/i})][\tilde{\U{r}}_{j/i}] - [\tilde{\U{r}}_{i/j}]
\label{eq:ytilde}
\end{equation}

The expected value of $\tilde{\B{y}}$, of course, is $\hat{\B{y}} = \B{0}$.

\section{ EKF measurement equations and noise model }

\subsection{Measurement vector and linearization}

The attitude coordinates $\B{q}_{j/i}$ are now specified as quaternion and re-designated $\BG{\beta}_{j/i} = \begin{bmatrix}
\beta_0 \\
\BG{\beta}
\end{bmatrix}$ for clarity. This allows the direction cosine matrix $C_{j/i}$ to be written as an explicit function of the coordinates:

\begin{equation}
[C_{j/i}(\BG{\beta}_{j/i})] = \B{I}_{3 \times 3} - 2\beta_0 \BG{\beta}^\times + 2\BG{\beta}^\times\BG{\beta}^\times
\end{equation}

The gradient of the measurement vector $\B{h}$ with respect to the attitude coordinates is now written in terms of $\hat{\BG{\beta}}_{j/i}$:

\begin{equation}
\ddarg{\B{h}}{\hat{\hat{\BG{\beta}}}_{j/i}} = \begin{bmatrix}
2\hat{\BG{\beta}}^\times [\tilde{\B{r}}_{j/i}] & 2\hat{\BG{\beta}}^\times[\tilde{\B{r}}_{j/i}^\times] + 2[\hat{\BG{\beta}}^\times [\tilde{\B{r}}_{j/i}]]^\times - 2\hat{\beta}_0 [\tilde{\B{r}}_{j/i}^\times]
\end{bmatrix}
\label{eq:HlinDef}
\end{equation}

\subsection{Error model and associated covariance matrix}

The pointing error and associated covariance matrix for each measurement are now considered. Each measured unit vector should be related to the true unit vector by a (hopefully) small rotation about an (unknown) rotation axis. If a reference frame is established whose 1 axis is parallel to the measured unit vector, the error vector to the true unit vector will lie in the 2-3 axis plane of that frame. In that reference frame, then, the measurement covariance is taken to be: $[R_e] = \mathrm{diag} \begin{bmatrix} 0 & \sigma_p^2 & \sigma_p^2
\end{bmatrix}$

The measurement covariance in the associated vehicle body frame is related to $[R_e]$ by a transformation matrix $[C_{b/e}]$:

\begin{equation}
[R_b] = [C_{b/e}][R_e][C_{b/e}]^T
\end{equation}

The cosine matrix $[C_{b/e}]$ is defined from the body frame unit vector measurement and one arbitrary orthogonal vector, chosen by computing the cross product of the measured unit vector with a unit vector whose components are chosen from the uniform distribution on $[-1,1]$. The body-frame covariance $[R_b]$ is independent of the choice of the orthogonal vector.

From the body-frame covariance $[R_b]$, the measurement covariance is now determined from the Jacobian $[J]$ of the output equation with respect to the unit vector. Defining the $6 \times 1$ vector $\tilde{\B{R}} = \begin{bmatrix}
\tilde{\B{r}}_{j/i} \\
\tilde{\B{r}}_{i/j}
\end{bmatrix}$, the Jacobian of $\B{h}$ is:

\begin{equation}
[J] \equiv \ddarg{\B{h}}{\tilde{\B{R}}} = \begin{bmatrix}
-[C_{j/i}] & -\B{I}_{3\times 3}
\end{bmatrix}
\end{equation}

The covariance matrix associated with $\tilde{\B{R}}$, in terms of agent $i$ and $j$'s body frame measurement covariances $[R_i]$ and $[R_j]$, respectively, is $[R_{\tilde{R}}]$:

\begin{equation}
\begin{bmatrix}
[R_{\tilde{R}}] = \begin{bmatrix}
[R_i] & \B{0}_{3\times 3} \\
\B{0}_{3\times 3} & [R_j]
\end{bmatrix}
\end{bmatrix}
\end{equation}

The covariance associated with the measurement vector $\B{h}$ is approximated by $[R_y]$:

\begin{equation}
[R_y] \equiv J[R_{\tilde{R}}]J^T
\end{equation}

The development in this section applies generally to the problem of relative attitude estimation, given the vector measurements between agents assumed. The EKF equations are now fully developed in each section for the particular cases considered.

\section{ First case: Relative attitude estimation with IMU shared }

In this scenario, agents are assumed to have access to current gyroscope measurements from one another. These measurements are used in state propagation. The estimated state is only the quaternion components $\BG{\beta}_{j/i}$, and an additive error is assumed for simplicity. The propagation in this case, derived from Eq. \ref{eq:omegaji}, reduces to:

\begin{align}
[A(\hat{\BG{\beta}}_{j/i})] \equiv \frac{1}{2} \begin{bmatrix}
-\hat{\BG{\beta}}^T \\
\hat{\beta}_0 \B{I}_{3\times 3} + \hat{\BG{\beta}}^\times
\end{bmatrix} \\
\dot{\hat{\BG{\beta}}}_{j/i} = [A(\hat{\BG{\beta}}_{j/i})] [\tilde{\BG{\omega}}_{j/n} - [C_{j/i}(\hat{\BG{\beta}}_{j/i})] \tilde{\BG{\omega}}_{i/n}] + 
\begin{bmatrix}
[A(\hat{\BG{\beta}}_{j/i})] & -[A(\hat{\BG{\beta}}_{j/i})][C_{j/i}(\hat{\BG{\beta}}_{j/i})]
\end{bmatrix} 
\begin{bmatrix}
\B{v}_i \\
\B{v}_j
\end{bmatrix} \\
\B{v}_i \sim N(0,R_{\omega,i}) \\
\B{v}_j \sim N(0,R_{\omega,j})
\end{align}

The gyroscope noise covariance $[R_{\omega}]$ is assumed to be the same for both agents. The linearized state influence matrix is given from:

\begin{align}
\BG{\omega}_p \equiv \tilde{\BG{\omega}}_{j/n}-\tilde{\BG{\omega}}_{i/n} \\
[F] \equiv \ddarg{\dot{\hat{\BG{\beta}}}_{j/i}}{\hat{\BG{\beta}}_{j/i}} = \frac{1}{2}
\begin{bmatrix}
0 & -\BG{\omega}_p^T \\ \BG{\omega}_p & -\BG{\omega}_p^\times
\end{bmatrix} + \begin{bmatrix}
\B{0}_{1\times 4}\\
2\hat{\beta}_0 \hat{\BG{\beta}}^\times \tilde{\BG{\omega}}_{i/n} & -\hat{\beta}_0^2 \tilde{\BG{\omega}}_{i/n}^\times + [\hat{\BG{\beta}}^\times \hat{\BG{\beta}}^\times \tilde{\BG{\omega}}_{i/n}]^\times + \hat{\BG{\beta}}^\times [\hat{\BG{\beta}}^\times \tilde{\BG{\omega}}_{i/n}]^\times + \hat{\BG{\beta}}^\times \hat{\BG{\beta}}^\times \tilde{\BG{\omega}}_{i/n}^\times 
\end{bmatrix}
\end{align}

The complicated expression for $[F]$ arises due to the product $[A(\hat{\BG{\beta}}_{j/i})][C_{j/i}(\hat{\BG{\beta}}_{j/i})]$. The preceding equations have fully defined the linearized state and process noise influence matrices for the EKF propagation step. Combined with the equations from the prior section, the full estimation problem has been outlined and relevant terms presented.

\subsection{Simulation}

In simulation, agents are initialized randomly in a 10 meter sphere. Each agent flies to a point in the sphere, chosen on the uniform distribution, following polynomial trajectories for position and Euler angles. Agents continue until the simulation time ends. Agents measure unit vectors to one another, and make IMU gyroscopic measurements in their own body frame. Using these two measurements only, both the Extended Kalman Filter (EKF) and Unscented Kalman Filter (UKF) are implemented using the equations developed in the previous section. Filter values are initialized with uniformly distributed initial quaternion components that are subsequently normalized.

Results are discussed briefly, as the primary purpose of this section is to establish the validity of the sequential filter for relative attitude determination in this framework. Results presented are for the UKF, whose performance is consistent with the EKF. Fig. \ref{fig:error_history_w_omega_measured} shows the error history for a 150 second simulation with two agents. The figure demonstrates that the filter is readily able to regulate the relative attitude estimate error. Fig. \ref{fig:estimate_history_w_omega_measured} shows the true and estimated relative attitude quaternion components for the first two seconds of simulation. Despite being initialized randomly, the filter is rapidly able to converge to the truth values.

\begin{figure}[tb!]
\centering
\includegraphics[width=1.0\textwidth]{error_history_w_omega_measured.png}
\caption{Relative heading estimate angular error history.}
\label{fig:error_history_w_omega_measured}
\end{figure}

\begin{figure}[tb!]
\centering
\includegraphics[width=1.0\textwidth]{estimate_history_w_omega_measured.png}
\caption{Relative quaternion estimate and truth for one agent in simulation.}
\label{fig:estimate_history_w_omega_measured}
\end{figure}

%\bibliographystyle{plain}

%\bibliography{refs}

\end{document}
